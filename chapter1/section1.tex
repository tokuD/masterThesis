\section{バリオンの構造}
ハドロンはクォークとグルーオンで構成される複合粒子であり、
3つのクォークから構成されるバリオンと1つのクォークと1つの反クォークから構成されるメソンに分類される。
構成子クォークモデルは、これまでに観測されている数多くのバリオンの性質を説明することができるが、
一部の励起状態やエキゾチックハドロンと呼ばれる状態を説明することは困難である。
近年、これらの状態を2つのクォーク間の相関であるダイクォーク相関を導入することで説明する試みがなされている\cite{ref1}。
しかしダイクォーク相関の存在は未だ実験的に確認できていない。

\subsection{チャーム・バリオンにおけるダイクォーク相関}
ダイクォーク相関とは2つのクォーク間の相関であり、3つのクォークで構成されるバリオンでは3対のダイクォーク相関が存在する。
アップクォークやダウンクォークのみからなる軽いバリオンでは、この3対のダイクォーク相関は縮退しており1対のみを分離すことは困難である。
しかし、軽いクォークの1つを重いクォークに置き換えることで軽いクォーク間のダイクォーク相関を運動学的に分離することが可能となる。
チャームバリオンは、アップクォーク、ダウンクォーク、ストレンジクォークらと比べて5倍程度の有効質量を持つチャームクォークを持つバリオンであるため、
ダイクォーク相関が顕在化すると考えられている。
図に示すように、チャームバリオンではダイクォークとチャームクォークの相対運動状態である$\lambda$モードと、
ダイクォークの内部励起状態である$\rho$モードに運動学的に分離し、アイソトープシフトと呼ばれる2つの励起モードが準位構造に現れる。
この2つの励起状態のエネルギー比は、
\begin{equation}
  \label{energy_ratio}
  \frac{\hbar\omega_{\rho}}{\hbar\omega_{\lambda}}=\sqrt{\frac{3m_{Q}}{2m_{q}+m_{Q}}}
\end{equation}
となる。
ここで$m_Q$と$m_q$はそれぞれ重いクォークと軽いクォークの構成子クォーク質量である。
重いクォークを含まず、3つのクォークが全て軽いクォークの場合は$m_Q=m_q$であり$(\ref{energy_ratio})$式より
励起状態のエネルギー比は1となる。つまり$\lambda$モードと$\rho$モードは縮退していることがわかる。
一方で、1つの重いクォークの質量$m_Q$が他のクォークの質量$m_q$に比べ十分に大きい場合、$(\ref{energy_ratio})$式は
\begin{equation}
  \label{energy_ratio2}
  \frac{\hbar\omega_{\rho}}{\hbar\omega_{\lambda}}\longrightarrow\sqrt{3}
\end{equation}
となる。
また、クォーク間のカラースピン相互作用はクォークの質量に反比例するため、軽いクォーク同士の相関は軽いクォークと重いクォークとの相関よりも強くなる。
したがって、チャームバリオンの励起エネルギーや生成率、崩壊率を詳細に測定することで、ダイクォーク相関を明らかにすることができると期待されている。
