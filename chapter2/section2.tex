\section{リングイメージング・チェレンコフ検出器}
$(\ref{eq:cherenkov_condition})$式から、荷電粒子の質量$m$と輻射体の屈折率$n$が既知の場合、
チェレンコフ放射が発生する荷電粒子の運動量の閾値$p_t$は$(\ref{eq:cherenkov_momentum_condition})$式のように表される。
\begin{equation}
  \label{eq:cherenkov_momentum_condition}
  p_t = \frac{m}{\sqrt{n^2-1}}
\end{equation}
したがって、荷電粒子の運動量が既知の場合、適切な屈折率の輻射体を用いることで荷電粒子の種類が特定できる。
このように、チェレンコフ放射の有無で粒子識別を行うものを閾値型チェレンコフ検出器という。
これに対し、チェレンコフ光を検出し、チェレンコフ角を測定することにより
$(\ref{eq:cherenkov_angle})$式から粒子の速度を求め、粒子識別を行うものをリングイメージング・チェレンコフ検出器という。
今回は、リングイメージング型のチェレンコフ検出器についての研究を行った。

\subsection{輻射体}
チャームバリオン分光実験では、リングイメージング・チェレンコフ検出器で2--16\space$\si{\GeV / c}$の広い
運動量領域での$\pi/K/p$の粒子識別が必要となるため、2種類の輻射体を用いる。
低運動量の粒子識別には低屈折率の輻射体


\subsection{球面鏡}

\subsection{光検出器}

\subsection{先行研究による要求性能}