\renewcommand{\abstractname}{概要}

\begin{abstract}
  構成子クォークモデルでは記述できないハドロンの励起状態を記述するために、
  新しい有効自由度としてダイクォーク相関が考えられているが、
  ダイクォーク相関は実験的に確認できていない。
  クォーク間のカラースピン相互作用はクォークの質量に反比例するため、
  バリオン中の1つのクォークをu, dと比べて重いチャームクォークにすることで軽いu, d間のダイクォーク相関を観測することができると考えられている。
  したがって、チャームバリオンの励起状態を包括的に測定することで、ダイクォーク相関の存在を明らかにすることができると期待されている。
  \\\indent
  % 我々は、J-PARC高運動量ビームラインにおいてチャームバリオン分光実験(J-PARC E50 実験)を計画している。
  % 実験では、液体水素標的に\SI{20}{\GeV\per c}の$\pi^{-}$ビームを入射し、
  % $\pi^{-} + p \rightarrow D^{*-} + Y_c^{*+}$反応によってチャームバリオンの励起状態$(Y_c^{*+})$を生成する。
  % $D^{*-}$の崩壊先である$\pi^-$, $K^+$と$\pi^{-}$ビームの四元運動量を測定することでミッシングマス法により$Y_c^{*+}$の質量を測定する。
  % $D^{*-}$の崩壊粒子は2-16 \si{\GeV / c}の広い運動量領域をもつ。\\
  % この広い運動量領域で粒子識別を行うためにリングイメージングチェレンコフ(RICH)検出器の開発を行なった。\\\indent
  % RICH検出器では、漏れ磁場の影響から光検出器として光電子増倍管(PMT)ではなくMPPCを使用する。
  % ただし、MPPCは受光面の面積が小さいため、チェレンコフ光を集光するためのコーン型ライトガイドを開発した。
  % 本研究では、東北大学電子光理学研究センター(ELPH)において検出面にコーン型ライトガイドとMPPCを使用したプロトタイプRICH検出器のテスト実験を行い、コーン型ライトガイドの性能評価を行なった。
  % また、Geant4によるシミュレーションを用いて実機における$\pi/K/p$の粒子識別性能を評価した。
  我々は、J-PARC高運動量ビームラインにおいてチャームバリオン分光実験(J-PARC E50 実験)を計画している。
  実験では、液体水素標的に\SI{20}{\GeV\per c}の$\pi^{-}$ビームを入射し、
  $\pi^{-} + p \rightarrow D^{*-} + Y_c^{*+}$反応によってチャームバリオンの励起状態$(Y_c^{*+})$を生成する。
  $D^{*-}$の崩壊先である$\pi^-$, $K^+$と$\pi^{-}$ビームの四元運動量を測定することでミッシングマス法により$Y_c^{*+}$の質量を測定する。
  $D^{*-}$の崩壊粒子は2$-$16 \si{\GeV / c}の広い運動量領域をもつ。
  この広い運動量領域で粒子識別を行うためにリングイメージングチェレンコフ(RICH)検出器の開発を行った。\\\indent
  RICH検出器では、漏れ磁場の影響から光検出器としてMulti-Pixel Photon Counter(MPPC)を使用する。
  MPPCの小さい面積の受光面で$\SI{2}{m}\times\SI{1}{m}$の検出面を覆う必要があるため、チェレンコフ光を集光するためのコーン型ライトガイドを開発した。
  コーン型ライトガイドを使用したプロトタイプ検出器の性能評価のため、
  東北大学電子光理学研究センター(ELPH)においてテスト実験を行った。
  $\SI{0.8}{GeV/c}$の陽電子を屈折率1.04のエアロゲルに照射して発生したチェレンコフ光を、
  曲率半径$\SI{3}{m}$の球面鏡で反射させ、コーン型ライトガイドとMPPCを使用してリングイメージを測定した。
  テスト実験の解析からコーンの集光性能や暗電流の影響を評価し、Geant4によるシミュレーションに実測値を反映させることで、
  実機における$\pi/K/p$の粒子識別性能を評価した。

\end{abstract}