\begin{thebibliography}{99}
  \bibitem{ref1} M.~Anselmino~${\it et\;al.}$, Rev. Mod. Phys. 65, 1199 (1993).
  \bibitem{ref2} H.~Noumi~${\it et\;al.}$, J-PARC P50 proposal, \\
  % http://www.j-parc.jp/researcher/Hadron/en/pac1301/pdf/P50_2012-19.pdf
  \bibitem{ref23} K.~Shirotori~${\it et\;al.}$, ``Charmed Baryon Spectroscopy Experiment at J-PARC``, JPS Conf. Proc. 8 022012 (2015).
  \bibitem{ref4} 山我拓巳, ``チャームバリオン分光実験用粒子識別検出器の設計``, 2013年度大阪大学理学研究科修士論文.
  \bibitem{ref5} 赤石貴也, ``チャームバリオン分光実験用ビームタイミング検出器の開発``, 2018年度大阪大学理学研究科修士論文.
  \bibitem{ref6} 辰巳凌平, ``低屈折率エアロゲルを用いた閾値型のエアロゲル・チェレンコフ粒子識別検出器の性能評価``, 2022年度大阪大学理学研究科修士論文.
  \bibitem{ref7} 国際商事株式会社, 技術資料
  \bibitem{ref8} 浜松ホトニクス株式会社, 技術資料/MPPC,
  % \\https://www.hamamatsu.com/resources/pdf/ssd/mppc_kapd9008j.pdf


  % \bibitem{ref10} S.~Agostinelliet~${\it et\;al.}$, “Geant4 A Simulation Toolkit”, Nuclear Instruments and Methods A 506 (2003) 250--303.
  % \bibitem{ref11} J.~Allison~${\it et\;al.}$, “Geant4 developments and applications”, IEEE Trans on Nuclear Science 53 No. 1 (2006) 270--278.
  % \bibitem{ref12} J.~Allison~${\it et\;al.}$, “Recent developments in Geant4”, Nuclear Instruments and Methods A 835 (2016) 186--225.

  % \bibitem{ref13} M.Tabata, "エアロゲルの開発と応用", http://www.jahep.org/hepnews/2019/19-4-3-aerogel.pdf
  % \bibitem{ref14} M.~Tabata~${\it et\;al.}$, "Hydrophobic silica aerogel production at KEK", Nuclear Instruments and Methods in Physics Research A 668 (2012) 64-70.

  % \bibitem{ref9} 浜松ホトニクス株式会社, MPPC arrays S13360 series, \\
  % https://www.hamamatsu.com/resources/pdf/ssd/s13360\_series\_kapd1052j.pdf
  % \bibitem{ref8} 浜松ホトニクス株式会社, MPPC arrays S13361-3050 series, \\
  %  https://www.hamamatsu.com/resources/pdf/ssd/s13361-3050\_series\_kapd1054e.pdf


  % \bibitem{ref15} SPring-8/実験施設/加速器, \\
  % http://www.spring8.or.jp/ja/about\_us/whats\_sp8/facilities/accelerators/
  % \bibitem{ref16} LEPS -- Laser Electron Photon Experiment at SPring-8, \\
  % http://www.rcnp.osaka-u.ac.jp/Divisions/np1-b/

  % \bibitem{ref18} Hadron Universal Logic Module, http://openit.kek.jp/project/HUL/HUL


  % \bibitem{ref17} T.~N.~Tomonori~${\it et\;al.}$, “Development of a FPGA-based high resolution TDC using Xilinx Spartan-6”, Annual Report 2016.
  % \bibitem{ref19} 汎用MPPC読み出しモジュール, https://openit.kek.jp/project/MPPC-Readout-Module/public/MPPC-Readout-Module

  % \bibitem{ref24} 浜松ホトニクス株式会社, private continuation

  % \bibitem{ref3} 小林和矢, "$\Sigma p$散乱実験用エアロゲルチェレンコフ検出器の開発", 2016年度大阪大学理学研究科修士論文.

\end{thebibliography}


